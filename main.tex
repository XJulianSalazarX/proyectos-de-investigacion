\documentclass{article}
\usepackage[utf8]{inputenc}
\usepackage[spanish]{babel}
\usepackage{listings}
\usepackage{graphicx}
\graphicspath{ {images/} }
\usepackage{cite}

\begin{document}

\begin{titlepage}
    \begin{center}
        \vspace*{1cm}
            
        \Huge
        \textbf{Memoria del computador}
            
        \vspace{0.5cm}
        \LARGE
        Taller
            
        \vspace{1.5cm}
            
        \textbf{Julian Ricardo Salazar Duarte}
            
        \vfill
            
        \vspace{0.8cm}
            
        \Large
        Departamento de Ingeniería Electrónica y Telecomunicaciones\\
        Universidad de Antioquia\\
        Medellín\\
        Septiembre de 2020
            
    \end{center}
\end{titlepage}

\tableofcontents

\newpage

\section{¿Qué es la memoria del computador?}
La memoria es una de las principales partes de un computador, en ella se almacena la información que se esta ejecutando en un computador de manera temporal, luego de dejar de usar un programa, la información sobre dicho programa es borrado o quitado de la memoria, para así, el espacio de memoria pueda ser usado para almacenar otra información otra información.
\vspace{0.5cm}

La memoria de una computadora trabaja en conjunto con el disco duro y un microprocesador, todos ellos en una placa madre (motherboard), al ejecutar un programa la memoria carga esa información, luego, el microprocesador procesa la información suministrada por la memoria, este la interpreta y le dice que debe hacer, por ejemplo, al abrir un archivo, la memoria recibe la información de querer abrir un archivo, este la pasa al microprocesador, y la información se borra de la memoria, el microprocesador toma la información o el archivo del disco duro (donde se almacena la información del computador) y lo carga en la memoria, para que este pueda ser usado, luego cuando se quiera cerrar dicho archivo, esté se elimina de la memoria, y si se han hecho cambios en un archivo son almacenados en el disco duro.


\section{Tipos de memoria} \label{contenido}
En un computador podemos encontrar distintos tipos de memoria, entre estos están el disco duro, la memoria RAM y la memoria cache.

\subsection{Disco Duro}
El disco duro es el lugar en el cual se almacena toda la información que se encuentra en una computadora, por lo general el disco duro tiene una capacidad del orden de los gigabytes y terabytes, está ubicado en la tarjeta madre, y podemos encontrar de dos tipos, las unidades de disco mecánico HDD (Hard Disk Drives) y las unidades de estado sólido o SSD (Solid State Drive)
\vspace{0.5cm}

La principal deferencia entre ambos tipos de discos es que el HDD (unidad de disco mecánico) funciona con una serie de discos que giran a alta velocidad, mientras un cabezal magnético escribe y lee datos en el disco, mientras que el SSD (unidad de estado solido) utiliza chips de memoria no volátil, y para escribir y leer datos se hace mediante impulsos eléctricos.\cite{andres2017cual} 

\subsection{Memoria RAM}
La memoria RAM (Random Access Memory) es la memoria principal de un computador, en donde se guarda de manera temporal los programas que se están ejecutando, está memoria  guarda la información, como su nombre lo indica, de forma aleatoria, a diferencia del disco solido que lo hace de manera secuencial, lo que hace a la memoria RAM más rápida.\cite{rebollo2011memoria}

\subsection{Memoria Caché}
La memoria caché se encuentra entre el procesador y la RAM o dentro del procesador, esta memoria se usa debido a que en la placa madre la RAM y el procesador están separados, lo que ralentiza al procesador, ya que si este es muy rápido, igual se me condicionado por la RAM, ya que solo puede procesar instrucciones al ritmo que la RAM lee la información.
\vspace{0.5cm}

Para solucionar dicho problema la memoria caché almacena los datos que más se estén utilizando, sin embargo, la memoria cache al tener mucho menos espacio que la RAM, solo puede guardar una cantidad limitada de datos.\cite{Salazar}

\section{¿Cómo se gestiona la memoria en un computador?}
La memoria es gestionada por el sistema operativo del computador, el cual se encarga de administrar espacios de memoria a los programas que están siendo ejecutados, así, como liberar los espacios de memoria que ya no están siendo utilizados; la parte del sistema operativo que se encarga de gestionar la memoria se conoce como administrador de memoria.
\vspace{0.5cm}

EL administrador de memoria tiende que cumplir los requisitos de reubicación (los programas deben ser cargados y descargados de la memoria), protección (proteger que los espacios de memoria que están siendo utilizados, por el sistema operativo o los programas que estén en ejecución no sean, para que no sean usados por otros programas), compartición (algunos procesos que deban compartir información, pueden acceder al mismo espacio de memoria), organización lógica (los programas se escriban como módulos) organización física (trasladar información entre la memoria principal y la secundaria).\cite{aguilera2015sistemas} 

\section{¿Qué hace que una memoria sea más rápida que otra?}
Para saber si una memoria es más rápida que otra, se mide por medio de dos factores, estos son la latencia y la frecuencia, por un lado la latencia es el tiempo que toma desde que el controlador de memoria envía una información  a la memoria de un computador y esta lee o escribe sobre un espacio de la memoria y envía la información y se mide en ciclos de reloj del sistema\cite{Salazar}; la frecuencia se mide en megahercios (MHz), y no es más que la velocidad a la que se trasportan los datos de la memoria.\cite{computerhoy}
\vspace{0.5cm}

Así, para calcular si una memoria es más rápida que otra se puede hacer mediante la siguiente formula: (latencia/frecuencia)*2*1024.\cite{computerhoy}

\bibliographystyle{IEEEtran}
\bibliography{references}

\end{document}


