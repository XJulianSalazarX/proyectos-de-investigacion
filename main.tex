\documentclass{article}
\usepackage[utf8]{inputenc}
\usepackage[spanish]{babel}
\usepackage{listings}
\usepackage{graphicx}
\graphicspath{ {images/} }
\usepackage{cite}

\begin{document}

\begin{titlepage}
    \begin{center}
        \vspace*{1cm}
            
        \Huge
        \textbf{Memoria del computador}
            
        \vspace{0.5cm}
        \LARGE
        Subtítulo
            
        \vspace{1.5cm}
            
        \textbf{Julian Ricardo Salazar Duarte}
            
        \vfill
            
        \vspace{0.8cm}
            
        \Large
        Departamento de Ingeniería Electrónica y Telecomunicaciones\\
        Universidad de Antioquia\\
        Medellín\\
        Septiembre de 2020
            
    \end{center}
\end{titlepage}

\tableofcontents

\section{¿Qué es la memoria del computador?}
La memoria es una de las principales partes de un computador, en ella se almacena la información que se esta ejecutando en un computador de manera temporal, luego de dejar de usar un programa, la información sobre dicho programa es borrado o quitado de la memoria, para así, el espacio de memoria pueda ser usado para almacenar otra información otra información.
\vspace{3pt}

La memoria de una computadora trabaja en conjunto con el disco duro y un microprocesador, todos ellos en una placa madre (motherboard), al ejecutar un programa la memoria carga esa información, luego, el microprocesador procesa la información suministrada por la memoria, este la interpreta y le dice que debe hacer, por ejemplo, al abrir un archivo, la memoria recibe la información de querer abrir un archivo, este la pasa al microprocesador, y la información se borra de la memoria, el microprocesador toma la información o el archivo del disco duro (donde se almacena la información del computador) y lo carga en la memoria, para que este pueda ser usado, luego cuando se quiera cerrar dicho archivo, esté se elimina de la memoria, y si se han hecho cambios en un archivo son almacenados en el disco duro.


\section{Tipos de memoria} \label{contenido}

En un computador podemos encontrar distintos tipos de memoria, entre estos están:
-El disco duro
-La memoria virtual
-La memoria RAM
-La memoria Cache



El paquete también agrega un comportamiento especial 
a <<estas marcas para hacer citas textuales>> tal como 
lo indican las reglas de la RAE. \cite{dirac}

\section{¿Cómo se gestiona la memoria en un computador?} 

\section{¿Qué hace que una memoria sea más rápida que otra?} 

A continuación se presenta el logo de C++ Figura (\ref{fig:cpplogo})
\begin{figure}[h]
\includegraphics[width=4cm]{cpplogo.png}
\centering
\caption{Logo de C++}
\label{fig:cpplogo}
\end{figure}

En la sección de teoremas (\ref{contenido})

\section{Conclusión} \label{conclulsion}

\bibliographystyle{IEEEtran}
\bibliography{references}

\end{document}

